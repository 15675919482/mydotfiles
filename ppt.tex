\documentclass[aspectratio=169,fontset=windows,UTF-8,10pt,xcolor={usenames,dvipsnames,svgnames,x11names}]{beamer}
\usepackage[space]{ctex}
\usetheme{cleancode}
%\usetikzlibrary{shadows}
\usepackage{tikz}
\usepackage{bookmark}
\usepackage{diagbox}
\usepackage[absolute,overlay,showboxes]{textpos}
\usepackage{indentfirst} 

\begin{document}
%首页
{
    \usebackgroundtemplate{\tikz[overlay,remember picture]\node[opacity=1]at (current page.center){\includegraphics[height=\paperheight,width=\paperwidth]{"/Users/shiyaoliang/z.我的资料/f.试验/ppt/jpg/z.jpg"}};}
    \begin{frame}[fragile]
        \begin{center}
        \Large 
            基于lsp和neovim的半隔离网络环境\\
                     c++开发环境搭建
            \\[4\baselineskip]
        \end{center}
    \end{frame}
}

%\frame 
%{
    %\frametitle{\secname}
    %\tableofcontents
    %%\tableofcontents
%}

%目录
\section[Contents]{}
\frame 
{
    \frametitle{\secname}
    \tableofcontents[pausesections]
    %\tableofcontents
}

\AtBeginSection[] 
{
    \frame<handout:0> 
    {
        \frametitle{Contents}
        \tableofcontents[current,currentsection]
    }
}

\usebackgroundtemplate{\tikz[overlay,remember picture]\node[anchor=45]at ([xshift=10ex,yshift=-5ex] current page.45) {\includegraphics[height=0.1\paperheight,width=0.2\paperwidth]{"/Users/shiyaoliang/z.我的资料/f.试验/ppt/jpg/z.jpg"}};}
\section{lsp简介}
\begin{frame}[fragile]
    \frametitle{lsp简介}
    \setlength{\parindent}{2em}
    近两年基于language server protocol(后文简称lsp)的language server和language client都在持续发展,已经到了可以使用的程度。
    这不仅使vscode的功能更加强大,也造福了其他功能可扩展的编辑器。
    对于vim而言,受益于lsp,可以拥有和vscode完全一样的语法检测和定义跳转等的体验,并且更加轻量。
    \\[1\baselineskip]
    现在ata上关于lsp介绍的文章还很少,本文旨在基于vim的lsp,在开发机的半网络隔离环境下,搭建C++开发环境。
\end{frame}

\section{vim VS neovim}
\begin{frame}[fragile]
\frametitle{vim VS neovim}
    \begin{table}[h]
        \centering
        \caption{vim VS neovim}
        \label{tab1}
        \begin{tabular}{|c|c|c|}
            \hline
            \diagbox{特性}{vim发行版} & neovim & vim \\
            \hline
            版本迭代    & 快     & 慢\\
            \hline
            支持feature & 多     & 少\\
            \hline
        \end{tabular}
    \end{table}
%目前除了vim外,还可以选择使用neovim,版本迭代更快,支持的fearture也更多,比如vim如果需要支持系统剪贴板,需要重新编译vim,如果需要映射meta键,mac上是option,windows上是alt,需要在vimrc文件里进行映射。而neovim的使用新一代的版本发行工具AppImage,可以直接下载可执行文件使用,天然支持系统剪贴板和meta键映射,配合emonade可以实现远程开发机和本地剪贴板互通,非常方便。我个人使用的是neovim,后面的介绍也是基于neovim。
\end{frame}

\section{nvimlsp VS coc.nvim}
\begin{frame}[fragile]
\frametitle{nvimlsp VS coc.nvim}
    \begin{table}[h]
        \centering
        \caption{nvimlsp VS coc.nvim}
        \label{tab2}
        \begin{tabular}{|c|c|c|c|}
            \hline
            \diagbox{特性}{lsp类型} & coc.nvim & nvimlsp & 其他\\
            \hline
            版本迭代     & 快     & 快    & 慢\\
            \hline
            支持feature  & 多     & 多    & 少\\
            \hline
            支持language & 多     & 多    & 少\\
            \hline
            稳定性       & 稳定   & 较稳定& 较稳定\\
            \hline
            补全体验     & 较快   & 快    & 较快\\
            \hline
            安装难度     & 较难   & 难    & 较难\\
            \hline
        \end{tabular}
    \end{table}
%在vim上可用的lsp主要是coc.nvim、vim-lsp等,coc.nvim是一个中国人主导开发的vim插件,实际是一个以lsp为主的插件平台,迭代速度、性能、支持的语言以及feature都大幅领先其他lsp插件。neovim除了coc.nvim外,0.5及以上版本还可以使用neovim内置的lsp,但目前而言,neovim内置的lsp还不成熟,补全的插件现在有comp,比前一代的compe性能和准确度有很大提升,但综合体验还不如coc.nvim,并且内置lsp的性能不如coc.nvim,index文件不显示进度,这些都是硬伤,因此现在更推荐coc.nvim。当然基于neovim的生态,以后neovim的内置lsp可能会超过coc.nvim,但以我个人的使用体验而言,至少0.7的dev版和coc.nvim的差距还比较大。
\end{frame}

\section{vim-plug VS packer.nvim}
\begin{frame}[fragile]
\frametitle{vim-plug VS packer.nvim}
    \begin{table}[h]
        \centering
        \caption{vim-plug VS packer.nvim}
        \label{tab3}
        \begin{tabular}{|c|c|c|}
            \hline
            \diagbox{特性}{包管理工具} & packer.nvim & vim-plug \\
            \hline
            使用语言             & lua       & vimscript\\
            \hline
            lazyload支持程度     & 高        & 低\\
            \hline
            是否内置             & 是        & 否\\
            \hline
            插件生态             & 较好      & 好\\
            \hline
            分块管理支持         & 好        & 较好\\
            \hline
        \end{tabular}
    \end{table}
%我个人使用vim插件管理工具vim-plug和packer.nvim。packer.nvim使用lua语言进行配置文件的编写,功能和vim-plug差不多,多了一些lazy load的功能,对neovim打开的速度有百ms级别的优化,可以提升对neovim的速度敏感的用户的体验。使用packer是以后的趋势,通常vim与emacs对比时,配置文件的语言vimscript是常被诟病的一点,而把配置文件语言改为lua无疑是为了补齐这块短板。
\end{frame}

\section{ccls VS clangd}
\hypersetup{colorlinks}
\begin{frame}[fragile]
    \frametitle{ccls VS clangd}
    \setlength{\parindent}{2em}
    c++的language server有两个,ccls和clangd。\\
    ccls是微软官方维护的C++ language server,是使用lsp进行c++开发的首选,clangd最新的版本对lsp的支持也不错。
    \\[1\baselineskip]
    ccls编译方法见:\\ 
    https://ata.alibaba-inc.com/articles/221681?spm=ata.25287382.0.0.1e2e45afqlKrPE
\end{frame}

%尾页
{
    \usebackgroundtemplate{\tikz[overlay,remember picture]\node[opacity=1]at (current page.center){\includegraphics[height=\paperheight,width=\paperwidth]{"/Users/shiyaoliang/z.我的资料/f.试验/ppt/jpg/z.jpg"}};}
    \begin{frame}
    \end{frame}
}
\end{document}
